\documentclass{article}
\usepackage{amsmath,amsthm}
\usepackage{amssymb,latexsym}
\usepackage{float}
\usepackage{fullpage}
\usepackage{times}


\newtheorem{theorem}{Theorem}
\newtheorem{corollary}[theorem]{Corollary}
\newtheorem{question}[theorem]{Question}
\newtheorem{lemma}[theorem]{Lemma}
\newtheorem{observation}[theorem]{Observation}
\newtheorem{proposition}{Proposition}
\newtheorem{definition}[theorem]{Definition}
\newtheorem{claim}[theorem]{Claim}
\newtheorem{fact}[theorem]{Fact}
\newtheorem{assumption}[theorem]{Assumption}
\newtheorem{example}{Example}
\newtheorem{conjecture}[theorem]{Conjecture}
\newtheorem{alg}[theorem]{Algorithm}

\newcommand{\set}[1]{{\left\{#1\right\}}}    % braces for set notation
\newcommand{\ve}[1]{\mathbf{#1}}
\newcommand{\abs}[1]{\left\lvert #1 \right\rvert}
\newcommand{\poly}{\operatorname{poly}}
\newcommand{\complex}{{\mathbb C}}
\newcommand{\reals}{{\mathbb R}}
\newcommand{\ints}{{\mathbb Z}}
\newcommand{\nats}{{\mathbb N}}
\newcommand{\proj}[1]{\mbox{$|#1\rangle \!\langle #1 |$}}
\newcommand{\enc}[1]{\left<#1\right>}
\newcommand{\spa}[1]{\mathcal{#1}}
\newcommand{\ayes}{A_{\rm yes}}
\newcommand{\ano}{A_{\rm no}}

\begin{document}

\title{
    CMSC 303 Introduction to Theory of Computation, VCU\\
    Assignment: 1\\
    Name: Steven Hernandez
}

\date{}

\maketitle
\vspace{-10mm}

\section{Exercises}
\begin{enumerate}
    \item % 1
        \begin{enumerate}
            \item $A \not\subseteq B$
            \item $A \subseteq B$
            \item $\set{x,y,z}$
            \item $\set{x,y}$
            \item $\set{(x,x),(x,y),(y,x),(y,y),(z,x),(z,y)}$
            \item $\set{\emptyset,\set{x},\set{y},\set{x,y}}$
        \end{enumerate}
    \item % 2
        $A \times B$ would simply have $a \times b$ elements because each element in $A$ will have to pair with each element of $B$. \\
        Let $A$ have $a$ elements, $B$ have $b$ elements and $n \in A$. \\
        From here we know that for just the one element $n$, we will have $S =\set{(n,B_0), (n,B_1),\ldots,(n,B_b)}$ such that $S \subseteq A \times B$ and where $\abs{S} = b$. \\
        Because we know this will happen for all elements in $A$, $\abs{A \times B} = ab$.
    \item % 3
        Let $C$ be a set such that $\abs{C}$ equals to some integer $c$. \\
        For each element in $C$, there are two options per element when building the sets contained within the power set; either the element appears or doesn't appear in the set. \\
        Thus there would be $2^c$ elements(sets) within $C$.
    \item % 4
        \begin{enumerate}
            \item $f(2) = 7$
            \item
                $domain = \mathbb{R}$ \\
                $codomain = \mathbb{R}$
            \item $g(2,10) = 9$
            \item
                $domain = \mathbb{R} \times \mathbb{R}$ \\
                $codomain = \mathbb{R}$
            \item $g(4,f(4)) = 8$
        \end{enumerate}
    \item % 5
        \begin{enumerate}
            \item $hasClassTogether$
            \item % TODO
            \item $canSeeFaceStraightOnWithoutMirror$ (or similar relective surface)
        \end{enumerate}
\end{enumerate}

\section{Problems}
\begin{enumerate}
    \item % 1
        The induction step is the problem. By removing the horse in sentence 2, the author assumes that $H_1$ is the same as the set assumed to be true in the first sentence. This however is not the case, the author can not make the assumption that $H_1$ nor $H_2$ contain horses of the same color from the information given.
    \item % 2
        \textbf{Claim:}
        \[
            \sum_{m=0}^n m = \frac{n(n+1)}{2}
        \]
        \begin{proof}
            \textbf{Base Case:} Prove for $n = 0$.
            \[
                \sum_{m=0}^n m = 0.
            \]

            \[
                \frac{0(0+1)}{2} = \frac{0}{2} = 0.
            \]

            \[
                0 = 0.
            \]

            \textbf{Induction Step:} For $k \geq 0$,  we assume
            \[
                \sum_{m=0}^k m = \frac{k(k+1)}{2}
            \]
            is true and want to prove that
            \[
                \sum_{m=0}^{k+1} m = \frac{(k+1)(k+1+1)}{2}
            \]
            is also true as a result.

            Note:
            \[
                \sum_{m=0}^{k+1} m = \sum_{m=0}^k m + (k + 1)
            \]

            \begin{align*}
                \sum_{m=0}^{k+1} m &= \frac{(k+1)(k+1+1)}{2} \\
                &= \frac{(k+1)(k+2)}{2} \\
                &= \frac{k(k+1) + 2(k+1)}{2} \\
                &= \frac{k(k+1)}{2} + \frac{2(k+1)}{2} &&\text{(by distributive property)}\\
                &= \frac{k(k+1)}{2} + (k+1) &&\text{(dividing by 2)}\\
                &= \frac{k(k+1)}{2} + (k+1).
                &&\qedhere
            \end{align*}

            Thus proving the claim.
        \end{proof}
\end{enumerate}

\end{document}
