\documentclass{article}
\usepackage{amsmath,amsthm}
\usepackage{amssymb,latexsym}
\usepackage{float}
\usepackage{fullpage}
\usepackage{times}

% graphs
\usepackage{tikz}
\usepackage{forest}
\usetikzlibrary{shadows,arrows.meta}

% grammars
\usepackage{listings}

\lstset{
  basicstyle=\itshape,
  xleftmargin=3em,
  literate={->}{$\rightarrow$}{2}
           {epsilon}{$\epsilon$}{1}
}

\newtheorem{theorem}{Theorem}
\newtheorem{corollary}[theorem]{Corollary}
\newtheorem{question}[theorem]{Question}
\newtheorem{lemma}[theorem]{Lemma}
\newtheorem{observation}[theorem]{Observation}
\newtheorem{proposition}{Proposition}
\newtheorem{definition}[theorem]{Definition}
\newtheorem{claim}[theorem]{Claim}
\newtheorem{fact}[theorem]{Fact}
\newtheorem{assumption}[theorem]{Assumption}
\newtheorem{example}{Example}
\newtheorem{conjecture}[theorem]{Conjecture}
\newtheorem{alg}[theorem]{Algorithm}

\newcommand{\set}[1]{{\left\{#1\right\}}}    % braces for set notation
\newcommand{\ve}[1]{\mathbf{#1}}
\newcommand{\abs}[1]{\left\lvert #1 \right\rvert}
\newcommand{\poly}{\operatorname{poly}}
\newcommand{\complex}{{\mathbb C}}
\newcommand{\reals}{{\mathbb R}}
\newcommand{\ints}{{\mathbb Z}}
\newcommand{\nats}{{\mathbb N}}
\newcommand{\proj}[1]{\mbox{$|#1\rangle \!\langle #1 |$}}
\newcommand{\enc}[1]{\left<#1\right>}
\newcommand{\spa}[1]{\mathcal{#1}}
\newcommand{\ayes}{A_{\rm yes}}
\newcommand{\ano}{A_{\rm no}}

\begin{document}

\title{
    CMSC 303 Introduction to Theory of Computation, VCU\\
    Assignment: 4\\
    Name: Steven Hernandez
}

\date{}

\maketitle
\vspace{-10mm}

\begin{enumerate}
    \item % 1
        \begin{enumerate}
            \item
                \begin{forest}
                    [E
                        [T
                            [F
                                [2]
                            ]
                        ]
                    ]
                \end{forest}
            \item
                \begin{forest}
                    for tree={
                        if n children=0{
                            font=\itshape,
                            tier=terminal,
                        }{},
                    }
                    [E
                        [E
                            [E
                                [T
                                    [F
                                        [2]
                                    ]
                                ]
                            ]
                            [+]
                            [T
                                [F
                                    [2]
                                ]
                            ]
                        ]
                        [+]
                        [T
                            [F
                                [2]
                            ]
                        ]
                    ]
                \end{forest}
            \item
                \begin{forest}
                    for tree={
                        if n children=0{
                            font=\itshape,
                            tier=terminal,
                        }{},
                        base=b
                    }
                    [E
                        [T
                            [T
                                [F
                                    [(]
                                    [E
                                        [E
                                            [T
                                                [F
                                                    [2]
                                                ]
                                            ]
                                        ]
                                        [+]
                                        [T
                                            [F
                                                [2]
                                            ]
                                        ]
                                    ]
                                    [)]
                                ]
                            ]
                            [\textit{$\times$}]
                            [F
                                [(]
                                [E
                                    [T
                                        [F
                                            [2]
                                        ]
                                    ]
                                ]
                                [)]
                            ]
                        ]
                    ]
                \end{forest}
        \end{enumerate}
    \item % 2
        \begin{enumerate}
            \item
            \item
        \end{enumerate}
    \item % 3
        \begin{enumerate}
            \item
                \[
                    S \rightarrow 0 | 1 | 0T0 | 1T1
                    T \rightarrow \epsilon | 1T | 0T
                \]

                Trivially $0$ and $1$ match.
                $0T0$ $1T1$ ensure that the first and last symbol are the same before moving past $S$ into $T$.
                $T$ simply allows you to add any symbols $\in \Sigma_{\epsilon}$ recursively within the string obtained above.
            \item
                \[
                    S_0 \rightarrow 0 | 1 | 0S_{odd} | 1S_{odd}
                    S_{odd} \rightarrow \epsilon | 0S_{even} | 1S_{even}
                    S_{even} \rightarrow 0 | 1 | 0S_{odd} | 1S_{odd}
                \]

                Think of the labels such that $S_{odd}$ means we have an odd length currently, thus we can only add $\epsilon$ of one symbol, which then means we now have an even number of symbols (thus the $S_{even}$).
            \item
                \[
                    S \rightarrow \epsilon | 0 | 1 | 0S0 | 1S1
                \]

                Unlike $a$, the only variable is $S$, this is because each time we recurse, we want to ensure whatever the sub-string contains, it always begins and ends with the same symbol, thus maintaining the palindrome.
            \item
                \[
                    S \rightarrow S
                \]

                The grammar continues recursively forever.
                Never reaching only terminals, thus never reaching an accept state.
            \item
                \begin{lstlisting}
                    S ->   X$C$X | C$X | X$C | C
                    C ->   1C1 | 0C0 | $ | $X$
                    X ->   $X | 1X | 0X | epsilon
                \end{lstlisting}

                The idea for this grammar is that $C$ always builds a palindrome.
                Note how from $C$, we either recursively wrap $C$ with $0$ or $1$.
                After which, we can leave $\$$ in the center.

                From the first step, if there are any symbols to the left or right of $C$, we delimite it with a $\$$.
                This keeps the grammar matching the language.

                This produces the language when $i = n$ and $j = n+1$, however does not account for $i = n$ and $j = n + t$ where $t > 1$.
                So, notice $C \rightarrow \$X\$$.
                This allows the palindrome we were building to have non-palindrome-like items in the middle of this palindrome.
                Notice though, that these symbols are delimited by $\$$ so that we keep separate the palindrome from earlier.
        \end{enumerate}
    \item % 4
        \begin{enumerate}
            \item
            \item
            \item
            \item
        \end{enumerate}
    \item % 5
\end{enumerate}

\end{document}
