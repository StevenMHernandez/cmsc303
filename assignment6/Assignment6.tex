
\documentclass{article}
\usepackage{amsmath,amsthm}
\usepackage{amssymb,latexsym}
\usepackage{epsfig}
\usepackage{hyperref}
\usepackage{float}
\usepackage{fullpage}
\usepackage{enumerate}
\usepackage{paralist}
\usepackage{times}


\newtheorem{theorem}{Theorem}
\newtheorem{corollary}[theorem]{Corollary}
\newtheorem{question}[theorem]{Question}
\newtheorem{lemma}[theorem]{Lemma}
\newtheorem{observation}[theorem]{Observation}
\newtheorem{proposition}{Proposition}
\newtheorem{definition}[theorem]{Definition}
\newtheorem{claim}[theorem]{Claim}
\newtheorem{fact}[theorem]{Fact}
\newtheorem{assumption}[theorem]{Assumption}
\newtheorem{example}{Example}
\newtheorem{conjecture}[theorem]{Conjecture}
\newtheorem{alg}[theorem]{Algorithm}

\newcommand{\myparagraph}[1]{\paragraph{#1.}}

\newcommand{\eps}{\varepsilon}
\newcommand{\epssdp}{\varepsilon_{\rm sdp}}

\newcommand{\C}{C}
\newcommand{\Tr}{Tr} %CHECK
\newcommand{\Id}{Id} %CHECK
\newcommand{\Exs}[2]{E_{#1}[#2]} %CHECK

\newcommand{\beq}{\begin{equation}}
\newcommand{\eeq}{\end{equation}}

\newcommand{\norm}[1]{\left\|\,#1\,\right\|}       % norm
\newcommand{\onorm}[1]{\norm{#1}_{\mathrm{1}}}      % Euclidean norm for vectors
\newcommand{\enorm}[1]{\norm{#1}_{\mathrm{2}}}      % Euclidean norm for vectors
\newcommand{\trnorm}[1]{\norm{#1}_{\mathrm {tr}}}  % trace norm
\newcommand{\fnorm}[1]{\norm{#1}_{\mathrm {F}}}    % frobenius norm
\newcommand{\snorm}[1]{\norm{#1}_{\mathrm {\infty}}}    % spectral norm

\newcommand{\set}[1]{{\left\{#1\right\}}}    % braces for set notation
\newcommand{\ve}[1]{\mathbf{#1}}
\newcommand{\abs}[1]{\left\lvert #1 \right\rvert}

\newcommand{\complex}{{\mathbb C}}
\newcommand{\reals}{{\mathbb R}}
\newcommand{\ints}{{\mathbb Z}}
\newcommand{\nats}{{\mathbb N}}
\newcommand{\rats}{{\mathbb Q}}

\newcommand{\proj}[1]{\mbox{$|#1\rangle \!\langle #1 |$}}
\newcommand{\enc}[1]{\left<#1\right>}

\newcommand{\spa}[1]{\mathcal{#1}}
\newcommand{\dens}{D(\spa{A}\otimes\spa{B})}
\newcommand{\unitaries}{U(\spa{A}\otimes\spa{B})}

\bibliographystyle{alpha}

\begin{document}

\title{
    CMSC 303 Introduction to Theory of Computation, VCU\\
    Assignment: 6\\
    Name: Steven Hernandez
}

\date{}
\maketitle

\vspace{-5mm}
\noindent 3.d uses http://cseweb.ucsd.edu/classes/sp06/cse105/homework8.pdf as reference

\noindent Total marks: $59$ marks + $6$ marks bonus for typing your solutions in LaTeX.\vspace{2mm}\\

\noindent Unless otherwise noted, the alphabet for all questions below is assumed to be $\Sigma=\set{0,1}$. This assignment will get you primarily to practice reductions in the context of decidability.

\begin{enumerate}
    \item {[10 marks]} We begin with some mathematics regarding uncountability. Let $\nats=\set{0, 1,2,3,\ldots}$ denote the set of natural numbers.
    \begin{enumerate}
        \item {[5 marks]}  Prove that the set of integers $\ints=\set{\ldots,-3,-2,-1,0,1,2,3\ldots}$ has the same size as $\nats$ by giving a bijection between $\ints$ and $\nats$.

                Proof: By bijection, we create a function $f: N \mapsto Z$ such that all even numbers in $N$ map into a some even number in $Z$ and odd numbers in $N$ map to negative numbers in $Z$.

                \[ f(x) = \begin{cases}
                  -\dfrac{x+1}{2}    & \text{if $x$ is odd}     \\
                  \dfrac{x}{2}       & \text{if $x$ is even}
               	\end{cases}
                \]

        \item {[5 marks]}  Let $B$ denote the set of all infinite sequences over $\set{0,1}$. Show that $B$ is uncountable using a proof by diagonalization.

                Proof: We can prove this by contradiction.

                Let's begin by only concern ourselves with very the large (infinite) strings.
                We assume $\exists$ list $L$ of all these strings.

                \begin{equation}
                    \begin{split}
                        L = & 011001100\ldots \\
                            & 100110011\ldots \\
                            & 100100100\ldots \\
                            & 111111111\ldots \\
                            & \ldots\ldots\ldots
                    \end{split}
                \end{equation}

                We can construct a string $x \in B$ which is not in $L$ by taking the $i$th symbol of x to be the opposite symbol from the $i$th entry of $L$. Thus, contradiction.

    \end{enumerate}
    \item {[9 marks]} We next move to a warmup question regarding reductions.
        \begin{enumerate}
            \item {[2 marks]} Intuitively, what does the notation $A\leq B$ mean for problems $A$ and $B$?

                    $A \leq B$ means that $B$ is harder than $A$ (or equally hard).

            \item {[2 marks]} What is a mapping reduction $A\leq_m B$ from language $A$ to language $B$? Give both a formal definition, and a brief intuitive explanation in your own words.

                            A mapping reduction gives us a way to handle deciding whether a problem is decidable or not from knowing that some other problem is decidable or not.

                            So for example with an example of adding and multiplying.
                            Multiplying $\leq$ adding because multiplying can be achieved by simply using adding. Thus, adding is a more powerful construct.

                            With this, we can say that if $A$ is not decidable, $B$ is not decidable either because $B$ is harder than $A$. As well as, if $B$ is decidable, $A$ is also decidable, because $A$ is not as hard as $B$.

            \item {[2 marks]} What is a computable function? Give both a formal definition, and a brief intuitive explanation in your own words.
            \item {[3 marks]} Suppose $A\leq_m B$ for languages $A$ and $B$. Please answer each of the following with a brief explanation.
                \begin{enumerate}
                    \item If $B$ is decidable, is $A$ decidable?

                            $A$ is decidable because $B$ is `harder`, yet is decidable.


                    \item If $A$ is undecidable, is $B$ undecidable?

                            $B$ is undecidable because $B$ is `harder` than $A$ and $A$ is not decidable.

                    \item If $B$ is undecidable, is $A$ undecidable?

                            It is unknown whether $A$ is decidable or not, because while $B$ is not decidable, it is also said to be harder than $A$.
                \end{enumerate}

        \end{enumerate}
    \item {[40 marks]} Prove using reductions that the following languages are undecidable.

    \begin{enumerate}
    		\item {[8 marks]} $L=\set{\enc{M}\mid M\text{ is a TM and }L(M)=\Sigma^*}$.

            		Proof: By contradiction. Assume $\exists$ TM $R$ deciding $L$.

                    If we can construct TM $S$ to decide $A_{TM}$ with $R$, this would be a contradiction.

                    First, given $\enc{M,x}$, we want to decide if $\enc{M,x} \in A_{TM}$. We will construct a new TM $M_x$ such that

                    If $x \in L(M)$, then $L(M_x) = \Sigma^*$ and \\
                    if $x \not\in L(M)$, then $L(M_x) \neq \Sigma^*$.

                    \begin{equation}
                        \begin{split}
                           M_x = "& \text{On input } t \in L(M): \\
                                  & \text{if } t = x, accept\\
                                  & \text{if } t \neq x, \text{run $M$ on $x$ and accept if $M$ does.}"
                        \end{split}
                    \end{equation}

                    This means if $M$ accepts, $L(M_x) = \Sigma^*$, and if $M$ does not accept, $L(M_x) = \Sigma^* - \set{x}$ thus $L(M_x) \neq \Sigma^*$.

                    Now we show how if we can create a TM $S$ from $R$ and $M_x$ which decides $A_{TM}$, we will have a contradiction.

                    \begin{equation}
                        \begin{split}
                           S = "& \text{On input} \enc{M,x} \text{for} A_{TM}: \\
                                & \text{construct TM } M_x \\
                                & \text{Run $R$ on $\enc{M_x}$} \\
                                & \text{If R accepts, (this means $L(M_x) = \Sigma^*$), accept} \\
                                & \text{If R rejects, reject.}"
                        \end{split}
                    \end{equation}

            \item {[8 marks]} $L=\set{\enc{M}\mid M\text{ is a TM and }\set{000,111}\subseteq L(M)}$.


            		Proof: By contradiction. Assume $\exists$ TM $R$ deciding $L$.

                    If we can construct TM $S$ to decide $A_{TM}$ with $R$, this would be a contradiction.

                    \begin{equation}
                        \begin{split}
                           S = "& \text{On input} \enc{M}: \\
                                &  \\
                        \end{split}
                    \end{equation}


            \item {[8 marks]} $L=\set{\enc{M}\mid M\text{ is a TM which accepts all strings of even parity}}$. (Recall the \emph{parity} of a string $x\in\set{0,1}$ is the number of $1$'s in $x$.)



        	\item {[8 marks]} $L=\set{\enc{M}\mid M\text{ is a TM that accepts }w^R\text{ whenever it accepts }w}$. Recall here that $w^R$ is the string $w$ written in reverse, i.e. $011^R=110$.

            Using http://cseweb.ucsd.edu/classes/sp06/cse105/homework8.pdf as reference:

            Proof: By contradiction. Assume $\exists$ TM $R$ deciding $L$.

                    If we can construct TM $S$ to decide $A_{TM}$ with $R$, this would be a contradiction.

                    We begin by creating a TM $M_x$ which defined as such:

                    \begin{equation}
                        \begin{split}
                           M_x = "& \text{On input} \enc{x}: \\
                                  & \text{If $x$ is not $01$ or $10$, reject} \\
                                  & \text{If $x$ is $01$, accept} \\
                                  & \text{Otherwise $x$ = $01$, so we run $M$ on $w$ and accept if $M$ accepts.} \\
                        \end{split}
                    \end{equation}

                    Note for this above machine, we could use any pairs $x$ and $y$ such that $x = y^R$. $L(M_x) = \set{01}$ if $M$ does not accept input $10$ or $L(M_x) = \set{01,10}$ if $M$ does accept input $10$. We can use this to show that building a TM $S$ which uses both $R$ and $M_x$, we would be able to decide $A_{TM}$, a contradiction.

                    \begin{equation}
                        \begin{split}
                           S = "& \text{On input} \enc{M,w}: \\
                                & \text{construct TM } M_x \\
                                & \text{Run $R$ on $\enc{M_x}$} \\
                                & \text{If R accepts, accept} \\
                                & \text{If R rejects, reject.}"
                        \end{split}
                    \end{equation}

        	\item {[8 marks]} Consider the problem of determining whether a TM M on an input $w$ ever attempts to move its head left when its head is on the left-most tape cell. Formulate this problem as a language and show that it is undecidable.
    \end{enumerate}
 \end{enumerate}
\end{document}
