\documentclass{article}
\usepackage{amsmath,amsthm}
\usepackage{amssymb,latexsym}
\usepackage{float}
\usepackage{fullpage}
\usepackage{times}

% graphs
\usepackage{tikz}
\usepackage[edges,linguistics]{forest}
\usetikzlibrary{automata,positioning,shadows,arrows.meta}

\tikzset{initial text={}}

% grammars
\usepackage{listings}

\lstset{
  basicstyle=\itshape,
  xleftmargin=3em,
  literate={->}{$\rightarrow$}{2}
           {epsilon}{$\epsilon$}{1}
}

\newtheorem{theorem}{Theorem}
\newtheorem{corollary}[theorem]{Corollary}
\newtheorem{question}[theorem]{Question}
\newtheorem{lemma}[theorem]{Lemma}
\newtheorem{observation}[theorem]{Observation}
\newtheorem{proposition}{Proposition}
\newtheorem{definition}[theorem]{Definition}
\newtheorem{claim}[theorem]{Claim}
\newtheorem{fact}[theorem]{Fact}
\newtheorem{assumption}[theorem]{Assumption}
\newtheorem{example}{Example}
\newtheorem{conjecture}[theorem]{Conjecture}
\newtheorem{alg}[theorem]{Algorithm}

\newcommand{\set}[1]{{\left\{#1\right\}}}    % braces for set notation
\newcommand{\ve}[1]{\mathbf{#1}}
\newcommand{\abs}[1]{\left\lvert #1 \right\rvert}
\newcommand{\poly}{\operatorname{poly}}
\newcommand{\complex}{{\mathbb C}}
\newcommand{\reals}{{\mathbb R}}
\newcommand{\ints}{{\mathbb Z}}
\newcommand{\nats}{{\mathbb N}}
\newcommand{\proj}[1]{\mbox{$|#1\rangle \!\langle #1 |$}}
\newcommand{\enc}[1]{\left<#1\right>}
\newcommand{\spa}[1]{\mathcal{#1}}
\newcommand{\ayes}{A_{\rm yes}}
\newcommand{\ano}{A_{\rm no}}

\begin{document}

\title{
    CMSC 303 Introduction to Theory of Computation, VCU\\
    Assignment: 6\\
    Name: Steven Hernandez
}

\date{}

\maketitle
\vspace{-10mm}

\begin{enumerate}
    \item % 1
        \begin{enumerate}
            \item
                Claim: $|N| = |Z|$ where $N = \set{0,1,2,3,\ldots}$ and $Z = \set{\ldots,-3,-2,-1,0,1,2,3,\ldots}$.

                Proof: By bijection, we create a function $f: N \mapsto Z$ such that all even numbers in $N$ map into a some even number in $Z$ and odd numbers in $N$ map to negative numbers in $Z$.

                \[ f(x) = \begin{cases}
                  -\dfrac{x+1}{2}    & \text{if $x$ is odd}     \\
                  \dfrac{x}{2}       & \text{if $x$ is even}
               	\end{cases}
                \]
            \item
        \end{enumerate}
        	Claim: $|N| \neq |B|$ where $B = \set{x | x \in \set{0,1}^*}$

            Proof: We can prove this by by contradiction.

            Suppose we only only concern ourselves with very the large (infinite) strings.
            We assume $\exists$ list $L$ of all thes strings.

            \begin{equation}
            	\begin{split}
            		L = & 011001100\ldots \\
                		& 100110011\ldots \\
                		& 100100100\ldots \\
                		& 111111111\ldots \\
                        & \ldots\ldots\ldots
                \end{split}
			\end{equation}

            We can construct the string $x \in B$ which is not in $L$ by taking the $i$th symbol of x to be the opposite symbol from the $i$th entry of $L$. Thus, contradiction.
    \item % 2
        \begin{enumerate}
            \item
            	$A \leq B$ means that $B$ is harder than $A$ (or equally hard).
            \item
            	A mapping reduction gives us a way to handle deciding whether a problem is decidable or not from knowing that some other problem is decidable or not.

                So for example with an example of adding and multiplying.
                Multiplying $\leq$ adding because multiplying can be achieved by simply using adding. Thus, adding is a more powerful construct.

                With this, we can say that if $A$ is not decidable, $B$ is not decidable either because $B$ is harder than $A$. As well as, if $B$ is decidable, $A$ is also decidable, because $A$ is not as hard as $B$.
            \item

            \item
              \begin{enumerate}
                  \item $A$ is decidable because $B$ is `harder`, yet is decidable.
                  \item $B$ is undecidable because $B$ is `harder` than $A$ and $A$ is not decidable.
                  \item It is unknown whether $A$ is decidable or not, because while $B$ is not decidable, it is also said to be harder than $A$.
              \end{enumerate}
        \end{enumerate}
    \item % 3
        \begin{enumerate}
            \item
            \item
            \item
        \end{enumerate}
\end{enumerate}

\end{document}
