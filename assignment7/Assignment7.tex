\documentclass{article}
\usepackage{amsmath,amsthm}
\usepackage{amssymb,latexsym}
\usepackage{epsfig}
\usepackage{hyperref}
\usepackage{float}
\usepackage{fullpage}
\usepackage{enumerate}
\usepackage{paralist}
\usepackage{times}
\usepackage{color}


\newtheorem{theorem}{Theorem}
\newtheorem{corollary}[theorem]{Corollary}
\newtheorem{question}[theorem]{Question}
\newtheorem{lemma}[theorem]{Lemma}
\newtheorem{observation}[theorem]{Observation}
\newtheorem{proposition}{Proposition}
\newtheorem{definition}[theorem]{Definition}
\newtheorem{claim}[theorem]{Claim}
\newtheorem{fact}[theorem]{Fact}
\newtheorem{assumption}[theorem]{Assumption}
\newtheorem{example}{Example}
\newtheorem{conjecture}[theorem]{Conjecture}
\newtheorem{alg}[theorem]{Algorithm}

\newcommand{\trace}{{\rm Tr}}

\newcommand{\norm}[1]{\left\|\,#1\,\right\|}       % norm
\newcommand{\onorm}[1]{\norm{#1}_{\mathrm{1}}}      % Euclidean norm for vectors
\newcommand{\enorm}[1]{\norm{#1}_{\mathrm{2}}}      % Euclidean norm for vectors
\newcommand{\trnorm}[1]{\norm{#1}_{\mathrm {tr}}}  % trace norm
\newcommand{\fnorm}[1]{\norm{#1}_{\mathrm {F}}}    % frobenius norm
\newcommand{\snorm}[1]{\norm{#1}_{\mathrm {\infty}}}    % spectral norm

\newcommand{\set}[1]{{\left\{#1\right\}}}    % braces for set notation
\newcommand{\ve}[1]{\mathbf{#1}}
\newcommand{\abs}[1]{\left\lvert #1 \right\rvert}

\newcommand{\complex}{{\mathbb C}}
\newcommand{\reals}{{\mathbb R}}
\newcommand{\ints}{{\mathbb Z}}
\newcommand{\nats}{{\mathbb N}}

\newcommand{\enc}[1]{\left<#1\right>}
\newcommand{\spa}[1]{\mathcal{#1}}

\newcommand{\ayes}{A_{\rm yes}} %CHECK
\newcommand{\ano}{A_{\rm no}} %CHECK

% tab
\newcommand\tab[1][1cm]{\hspace*{#1}}

\bibliographystyle{alpha}

\begin{document}

\title{CMSC 303 Introduction to Theory of Computation, VCU\\Spring 2017, Course Mini-Project\\Due: Tuesday, May 2, 2017 in class \\ Steven Hernandez}
\date{}
\maketitle
\vspace{-5mm}
\noindent Total marks: $26$ marks + $3$ marks bonus for typing your solutions in LaTeX.\vspace{2mm}\\

One of the goals of this course is to train you to be able to expand your education independently, given the tools you've learned here. For this reason, your mini-project will consist of the following task. First, please independently read Sections 8.1 and 8.2 on Space Complexity. This will introduce you to the complexity class PSPACE and Savitch's theorem. Next, answer the following questions.

\begin{enumerate}
    \item {[2 marks]} In words, which class of problems does PSPACE characterize?

    		%1
            $\textcolor[RGB]{220,220,220}{\rule{\linewidth}{0.2pt}}$
            PSPACE is meant to characterize all problems which can be completed on a TM using only $O(f(n))$ cells of the TM where $f(n)$ is a function that can of polynomial time. Savitch’s theorem shows that NP $\subseteq$ PSPACE.

            $\textcolor[RGB]{220,220,220}{\rule{\linewidth}{0.2pt}}$

    \item {[4 marks]} Show that PSPACE is closed under the concatenation and star operations.

    		%2
            $\textcolor[RGB]{220,220,220}{\rule{\linewidth}{0.2pt}}$
            First for concatenation:

            Suppose we have two TMs which run in $O(f(n)) \subseteq$ PSPACE, labeled $A$ and $B$ for input(s) $n$. We construct a second TM $C$ which simulates both $A$  and $B$. Our goal is to run $C$ such that $C \subseteq$ PSPACE. We can input into $C$ the input for $A$ and $B$ separated by a delimiter $d \not\in \Sigma$. For this example we will simply use a comma $,$

            \begin{equation}
            	\begin{split}
            		C = "& \text{On input $m$} \\
            			 & \text{Split $m$ on $,$ to retrieve inputs $a$ and $b$} \\
            			 & \text{Run $A$ on $a$ in PSPACE} \\
                         & \text{Reset the tape by clearing the content.} \\
            			 & \text{Return pointer back to the first cell} \\
                         & \text{Run $B$ on $b$ in PSPACE overwriting anything written by $A$"}
            	\end{split}
            \end{equation}

            Because we move the pointer back to the first cell and overwrite any values, we are back to a clear TM, thus if $B$ ran in PSPACE normally, it would run in PSPACE this time as well, because there are no remnants from $A$. The amount of space used would be the greater of the space used by $A$ and $B$.

            Because of this, showing $^*$ is trivial because we can run as many TMs (that are $\subseteq$ PSPACE) on this machine as we wanted. Because each machine runs overtop of the same cells as the previous machine, completely unaware of the past.

            $\textcolor[RGB]{220,220,220}{\rule{\linewidth}{0.2pt}}$

    \item {[4 marks]} Prove that ${\rm NP}\subseteq {\rm PSPACE}$ by giving a polynomial-space algorithm for simulating an NP machine (use Theorem 7.20 for this, which says a language is in NP iff it is decided by a non-deterministic polynomial time Turing machine).

            %3
            $\textcolor[RGB]{220,220,220}{\rule{\linewidth}{0.2pt}}$
            Theorem 7.20 shows us that a language in NP must be decidable by nondeterministic polynomial time TM. This means that if the time the TM takes is $O(f(n))$, where $f(n)$ takes polynomial time, then there is no way for the TM to use more than polynomial space. If a polynomial time TM spends every step moving left and writing a character, the furthest it could reach would be equal to the amount of time it spent, which itself is polynomial.

            So what we can do is we can create a TM $S$ that models a Breadth First Search of the NTM.

            \begin{equation}
            	\begin{split}
            		S = "& \text{On input $n$} \\
            			 & \text{Starting from the first configuration} \\
            			 & \text{If the configuration is accepting, accept} \\
                         & \text{Search each configuration as a BFS, erasing the tape and returning to the first cell after each.} \\
            			 & \text{accept once an accepting configuration is reached.}
            	\end{split}
            \end{equation}

            $\textcolor[RGB]{220,220,220}{\rule{\linewidth}{0.2pt}}$

            \item {[2 marks]} Why is Savitch's theorem surprising, given our study of P versus NP?

            %4
            $\textcolor[RGB]{220,220,220}{\rule{\linewidth}{0.2pt}}$
            The book made a great statement for Example 8.3 "Space appears to be more powerful than time because space can be reused, whereas time cannot."

            But what this means is, we can model the hard problems on physical computers, because computers are limited when it comes to memory size, but in essence, could run forever.

            But what else this could mean is, there may be more problems. What sort of problems might require exponential space to compute? This implies a whole new set of problems.

            $\textcolor[RGB]{220,220,220}{\rule{\linewidth}{0.2pt}}$

            \item {[14 marks]} This question studies the proof of Savitch's theorem. You may assume the TM $M$ in the proof can compute function $f(n)$ in the proof in $O(f(n))$ space.
            \begin{enumerate}
                \item {[2 marks]} In words, give a high-level overview of how the proof of Savitch's theorem works.

                %5.a
            $\textcolor[RGB]{220,220,220}{\rule{\linewidth}{0.2pt}}$

            $\textcolor[RGB]{220,220,220}{\rule{\linewidth}{0.2pt}}$

                \item {[6 marks]} The CANYIELD procedure in the proof has 6 steps: Describe in a sentence or two the purpose of each step.

                %5.b
            $\textcolor[RGB]{220,220,220}{\rule{\linewidth}{0.2pt}}$

            $\textcolor[RGB]{220,220,220}{\rule{\linewidth}{0.2pt}}$

                \item {[2 marks]} Why does the machine $M$ in the proof correctly simulate the NTM $N$?

                %5.c
            $\textcolor[RGB]{220,220,220}{\rule{\linewidth}{0.2pt}}$

            $\textcolor[RGB]{220,220,220}{\rule{\linewidth}{0.2pt}}$

                \item {[4 marks]} Why does $M$ use $O(f^2(n))$ space?

                %5.d
            $\textcolor[RGB]{220,220,220}{\rule{\linewidth}{0.2pt}}$

            $\textcolor[RGB]{220,220,220}{\rule{\linewidth}{0.2pt}}$

            \end{enumerate}
\end{enumerate}

\end{document}